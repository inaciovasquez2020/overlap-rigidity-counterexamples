\documentclass[11pt]{article}

\usepackage[T1]{fontenc}
\usepackage[utf8]{inputenc}
\usepackage[margin=1in]{geometry}
\usepackage{amsmath,amssymb,amsthm,mathtools}
\usepackage{hyperref}
\usepackage{enumitem}

\newtheorem{theorem}{Theorem}
\newtheorem{lemma}[theorem]{Lemma}
\newtheorem{proposition}[theorem]{Proposition}
\newtheorem{corollary}[theorem]{Corollary}
\newtheorem{definition}[theorem]{Definition}
\newtheorem{claim}[theorem]{Claim}

\newcommand{\F}{\mathbb{F}}
\newcommand{\E}{\mathbb{E}}
\newcommand{\Prb}{\mathbb{P}}
\newcommand{\bits}{\mathrm{bits}}
\newcommand{\I}{\mathrm{I}}
\newcommand{\Hh}{\mathrm{H}}
\newcommand{\TC}{\mathrm{TC}}

\newcommand{\Ball}{\mathrm{Ball}}
\newcommand{\rank}{\mathrm{rank}}
\newcommand{\im}{\mathrm{im}

}

\title{UCLP$(\mathcal P)$ for an Explicit Constant-Width SAT Family:\\
Local Conditioning Bounds and a Linear Transcript-Capacity Lower Bound}
\author{}
\date{}

\begin{document}
\maketitle

\section{Objects and conventions}

All logarithms are base $2$. Entropies and mutual informations are measured in $\bits$.
A \emph{CNF} formula $F$ is a conjunction of clauses over Boolean variables. A \emph{width-$w$ CNF}
has every clause of size at most $w$.

Let $X$ be a random variable uniformly distributed over the satisfying assignments of $F$.
For any random variable $Y$, define Shannon entropy $\Hh(Y)$ and mutual information $\I(X;Y)$
in the standard way. For a filtration (transcript) $(S_t)_{t=0}^m$, define the \emph{transcript capacity}
\[
\TC(S_{0:m}) \;:=\; \sum_{t=1}^m \I(X;S_t \mid S_{t-1}).
\]

\subsection{Canonical $r$-local transcripts}
Fix an integer radius $r\ge 1$. A \emph{canonical $r$-local transcript} for inputs of size $n$ is a sequence
$T=(S_0,S_1,\dots,S_m)$ where each transition $S_{t-1}\to S_t$ is computed from the current state by
inspecting only an $r$-neighborhood in a bounded-degree representation graph of the state.
The only property used in this document is the following locality-to-conditioning principle.

\begin{definition}[Local conditioning proxy]
Fix a bounded-degree bipartite factor graph representation $\mathcal G(F)$ of a CNF $F$
(variables--clauses incidence with bounded degrees).
Let $\Ball_{\mathcal G}(u,r)$ denote the radius-$r$ ball around a node $u$ in $\mathcal G(F)$.
A transcript step is \emph{$r$-local} if, conditioned on the prior state, its new information is measurable
with respect to the sigma-algebra generated by the labels/constraints within some radius-$r$ ball
$\Ball_{\mathcal G}(u,r)$.
\end{definition}

Accordingly, it suffices to bound the mutual information about $X$ revealed by conditioning on any fixed
radius-$r$ ball in $\mathcal G(F)$.

\section{An explicit constant-width SAT family from LDPC/Tseitin constraints}

\subsection{LDPC parity-check instances}
Fix constants $\Delta_v,\Delta_c\ge 3$. For each $n$, let $H_n$ be a bipartite graph with:
\begin{itemize}[leftmargin=2em]
\item variable nodes $V_n$ with $|V_n|=n$,
\item check nodes $C_n$ with $|C_n|=\Theta(n)$,
\item every $v\in V_n$ has degree $\Delta_v$, every $c\in C_n$ has degree $\Delta_c$,
\item girth $\mathrm{girth}(H_n)\ge 4r+4$.
\end{itemize}
Such families exist (explicit constructions are known) and can be taken to be expanders; expansion is not used
for the linear lower bound, only bounded degree and girth.

Let $A_n\in \F_2^{|C_n|\times n}$ be the incidence matrix of $H_n$ (row for each check, column for each variable).
Fix a right-hand side $b_n\in \F_2^{|C_n|}$ such that the linear system
\[
A_n x = b_n
\quad\text{over }\F_2
\]
is consistent. Define $\mathsf{Sol}_n:=\{x\in \F_2^n : A_n x=b_n\}$.

\begin{proposition}[Solution space size]
If $A_n x=b_n$ is consistent then $\mathsf{Sol}_n$ is an affine subspace of $\F_2^n$ of size
$|\mathsf{Sol}_n|=2^{n-\rank(A_n)}$.
\end{proposition}

\subsection{Constant-width Tseitin-to-CNF encoding}
We encode each check equation (a parity constraint on $\Delta_c$ bits) using a standard constant-size CNF gadget
with auxiliary variables.

Fix $\Delta_c$. For a check node $c\in C_n$ with neighbors $N(c)=\{v_1,\dots,v_{\Delta_c}\}$, introduce auxiliary
variables $y_{c,1},\dots,y_{c,\Delta_c-1}$ and enforce:
\[
y_{c,1} = x_{v_1}\oplus x_{v_2},\quad
y_{c,2} = y_{c,1}\oplus x_{v_3},\quad \dots,\quad
y_{c,\Delta_c-1} = y_{c,\Delta_c-2}\oplus x_{v_{\Delta_c}},
\]
and finally $y_{c,\Delta_c-1}=b_{n}(c)$.
Each XOR relation $z=u\oplus v$ can be expressed by a width-$3$ CNF using four clauses:
\[
(z\vee u\vee v)\wedge(z\vee \neg u\vee \neg v)\wedge(\neg z\vee u\vee \neg v)\wedge(\neg z\vee \neg u\vee v).
\]
Also $z=b$ is enforced by a unit clause.

Let $F_n$ be the conjunction of all such gadget clauses over all checks $c\in C_n$.

\begin{proposition}[Width, bounded degree, and satisfiable assignments]
Each $F_n$ is a width-$3$ CNF.
Moreover, there is a bijection between $\mathsf{Sol}_n$ and satisfying assignments of $F_n$ restricted to the
original variables $x_v$:
every $x\in\mathsf{Sol}_n$ extends uniquely to a satisfying assignment of $F_n$ over all variables
(including the auxiliaries), and every satisfying assignment restricts to an $x\in\mathsf{Sol}_n$.
\end{proposition}

\begin{proof}
Width-$3$ is immediate from the XOR CNF encoding above.
Given $x\in\mathsf{Sol}_n$, the sequential definitions force each $y_{c,j}$ uniquely, and the gadget clauses
are satisfiable. Conversely, the XOR gadgets enforce the parity equalities, hence the restriction $x$ must satisfy
$A_n x=b_n$.
\end{proof}

\subsection{Factor graph and locality}
Let $\mathcal G(F_n)$ be the variable--clause incidence graph of the CNF $F_n$.
Because $H_n$ has bounded degrees and each check contributes a constant-size gadget, $\mathcal G(F_n)$ has bounded degree
(independent of $n$), and the girth condition on $H_n$ implies that radius-$r$ neighborhoods in $\mathcal G(F_n)$
are acyclic (trees) after possibly increasing $r$ by a constant depending only on $\Delta_c$.

\section{Local conditioning reveals only constant information}

Let $X$ be uniform over satisfying assignments of $F_n$ (equivalently, uniform over $\mathsf{Sol}_n$ pushed forward through
the unique extension map).

\subsection{Linear-algebraic view}
Because satisfying assignments correspond to an affine $\F_2$-subspace on the original variables, $X$ induces an affine
distribution on those variables, and the auxiliary variables are deterministic functions of the originals.
Hence mutual information about $X$ revealed by any local ball is bounded by the number of independent linear constraints
that the ball imposes.

\begin{lemma}[Tree-local constraints have constant rank]\label{lem:local-rank}
Fix $r\ge 1$. There exists a constant $R=R(r,\Delta_v,\Delta_c)$ such that for every $n$, and every radius-$r$ ball
$B$ in the factor graph $\mathcal G(F_n)$, the set of parity constraints on the original variables implied by the clauses
inside $B$ has $\F_2$-rank at most $R$.
\end{lemma}

\begin{proof}
Because $\mathcal G(F_n)$ has bounded degree, the ball $B$ contains at most $O(1)$ variables and clauses.
Each XOR-gadget clause set inside $B$ enforces a constant number of parity relations among the variables present in $B$.
All such relations live in a vector space over $\F_2$ of dimension at most the number of original variables in $B$.
Therefore the rank is bounded by a constant $R$.
\end{proof}

\begin{lemma}[Local conditioning leaks $O(1)$ bits]\label{lem:local-info}
Fix $r\ge 1$ and let $B$ be any radius-$r$ ball in $\mathcal G(F_n)$.
Let $Z_B$ be the complete induced labeled sub-instance inside $B$.
Then
\[
\I(X; Z_B)\ \le\ R(r,\Delta_v,\Delta_c).
\]
\end{lemma}

\begin{proof}
Auxiliary variables are deterministic functions of original variables under satisfiable assignments,
so $Z_B$ is a deterministic function of a constant-sized restriction of $X$.
Thus $\Hh(Z_B)\le R$ and $\I(X;Z_B)\le \Hh(Z_B)\le R$.
\end{proof}

\subsection{Per-step bound}

\begin{corollary}[Per-step capacity bound]\label{cor:perstep}
Fix $r\ge 1$. For any canonical $r$-local transcript $T=(S_t)_{t=0}^m$ operating on $F_n$ and any $t\ge 1$,
\[
\I(X; S_t \mid S_{t-1}) \le R(r,\Delta_v,\Delta_c).
\]
\end{corollary}

\begin{proof}
By $r$-locality, conditioned on $S_{t-1}$, the new information in $S_t$ is measurable with respect to some $Z_B$.
Hence $\I(X;S_t\mid S_{t-1})\le \I(X;Z_B)\le R$.
\end{proof}

\section{A linear transcript-capacity lower bound}

\begin{lemma}[Entropy requirement]\label{lem:entropyreq}
Let $X$ be uniform over satisfying assignments of $F_n$. Then
\[
\Hh(X)=\log_2 |\mathsf{Sol}_n| = n-\rank(A_n).
\]
\end{lemma}

\begin{theorem}[Linear lower bound on transcript capacity]\label{thm:linearTC}
Fix $r\ge 1$. Assume $\rank(A_n)\le (1-\gamma)n$ for some constant $\gamma>0$.
Let $T=(S_t)_{t=0}^m$ be any canonical $r$-local transcript that determines $X$ uniquely.
Then
\[
\TC(T)\ \ge\ \gamma n.
\]
\end{theorem}

\begin{proof}
If $X$ is determined by $S_m$, then $\Hh(X\mid S_m)=0$ and
\[
\TC(T)=\I(X;S_m)=\Hh(X)\ge \gamma n.
\]
\end{proof}

\begin{corollary}[UCLP$(\mathcal P)$ is NO]\label{cor:uclpno}
Let $\mathcal P=\{F_n\}$ be the SAT family above.
Then any fixed-radius local transcript solving $\mathcal P$ must have transcript capacity $\Omega(n)$.
\end{corollary}

\section{Remarks on $\Omega(n\log n)$ strengthening}

The present argument establishes an unconditional linear lower bound.
An $\Omega(n\log n)$ bound requires additional information-diffusion or entropy-decay mechanisms not assumed here.

\end{document}

